\documentclass[12pt,utf8]{beamer}

% Gute Einführung zu LaTeX-Beamer: http://www2.informatik.hu-berlin.de/~mischulz/beamer.html

%-----PARAMETERS-----

%Wichtige Standard Pakete!
\usepackage[ngerman]{babel}
\usepackage{xcolor}
\usepackage{graphicx}
\usepackage{tikz}

%Für den Header notwendig!
\usepackage[percent]{overpic}

%Einbinden des Themes
\input{beamerthemeTU.sty}

%HintergrundLogo
%\setbeamertemplate{background}{\tikz[remember picture, overlay]\node[opacity=0.4] at (current page.center) {\includegraphics[width=2cm]{res/briar_logo_large.png}};}

%Standard Angaben
\title{Brumble und Briar}
\subtitle{Der (vielleicht) nächste Schritt in der Messeging-Evolution}

\author[C. Parnitzke]{Christoph Parnitzke}
\institute[FOSS AG - FbI]{Free and Open Source Software AG\\ Fachbereich Informatik}

\date{\today}

%-----IMPLEMENTATION-----

\begin{document}

\begin{frame}
	\titlepage
\end{frame}

\begin{frame}[TU]{Inhaltsverzeichnis}
	\tableofcontents[hideallsubsections]
\end{frame}

\section{Motivation}

\begin{frame}{Inhaltsverzeichnis}
	\tableofcontents[currentsection, hideallsubsections]
\end{frame}


%-----Motivation - Einleitung-----
\begin{frame}{Motivation}{Schon wieder ein neuer Messenger?!}
	\begin{center}
	\includegraphics[scale=0.3]{res/briar_logo_large.png}
	\end{center}
\end{frame}

\begin{frame}{Motivation}{Rückblick}
	\begin{block}{Was es bereits gibt}
		\begin{itemize}
			\item WhatsApp
			\item Telegram
			\item Threema
			\item Signal
			\item XMPP (Jabber)
		\end{itemize}
	\end{block}
\end{frame}

%WhatsApp
\subsection{WhatsApp}
\begin{frame}{Motivation}{WhatsApp}
	\begin{block}{Der Pop-Stern}
		\begin{itemize}[<+->]
			\item Über 1 Mrd. Nutzer weltweit
			\item Featurefull (Gruppen, Voicemessages, Dateiversand)
			\item Leichte, intuitive Bedienung
			\item Kostenlos
			\item \color{green} Implementiertes Axolotl (Crypto)
			\item \color{red} non-anonym, Closed Source, Metadata-Mining, Zentralisiert
		\end{itemize}
	\end{block}
\end{frame}

%Telegram
\subsection{Telegram}
\begin{frame}{Motivation}{Telegram}
	\begin{block}{The russian expierience}
		\begin{itemize}[<+->]
			\item Über 500 Mill. Nutzer
			\item Featurefull (Weiterleitungen, Kanäle, Sticker, Bots)
			\item Leichte Bedienung
			\item Kostenlos
			\item Eigene E2E-Crypto
			\item \color{green} Open Source (Ermöglicht bspw. eig. Bots und Sticker)
			\item \color{red} non-anonym, Crypto ist kein Standard, Data-Mining, Zentralisiert
		\end{itemize}
	\end{block}
\end{frame}

%Threema
\subsection{Threema}
\begin{frame}{Motivation}{Threema}
	\begin{block}{Stark wie die Alpen}
		\begin{itemize}[<+->]
			\item Viele Mill. User
			\item Featurefull (Gruppen, Bestätigungen)
			\item Leichte Bedienung
			\item Kostenpflichtig
			\item Crypto-Audit, optional auch anonym nutzbar
			\item \color{green} Privatsphäre-kritische Features optional, Crypto entspricht Standards
			\item \color{red} Closed Source, Metadata-Mining, Zentralisiert
		\end{itemize}
	\end{block}
\end{frame}

%Signal
\subsection{Signal}
\begin{frame}{Motivation}{Signal}
	\begin{block}{Die (fast) freie Alternative}
		\begin{itemize}[<+->]
		\item \color{green} Nutzerzahlen unbekannt?
		\item Featurefull (Gruppen, Sprachnachrichten, etc.)
		\item Leichte Bedienung
		\item Kostenfrei
		\item \color{green} Freier Code (Crypto, Client), sichere und moderne Crypto
		\item \color{red} Non-anonym, basiert auf nicht-freier Infrastruktur (Google), Metadata-Mining, Quasi-Zentralisiert
		\end{itemize}
	\end{block}
\end{frame}

%XMPP
\subsection{XMPP}
\begin{frame}{Motivation}{XMPP}
	\begin{block}{Der Altmeister}
		\begin{itemize}[<+->]
		\item \color{green} Nutzerzahlen unbekannt
		\item Featurefull (jedes Feature ist optional)
		\item Bedienung je nach Client (komfortabel bis kompliziert)
		\item kostenfrei
		\item \color{green} anonym, freier Code, versch. Crypto möglich, verteilte Server
		\item \color{red} Geringe Standardisierung, viele Baustellen, Crypto ist optional
		\end{itemize}
	\end{block}
\end{frame}

%-----Motivation - Übergang zum Thema-----
\begin{frame}{Motivation}{Was haben alle gemein?}
	\begin{itemize}
	\item Abhängig von fragiler Infrastruktur (Internet, Server)
	\item Keine einheitliche, einfach zu implementierende Crypto
	\item Keine Anonymität gewährleistet
	\end{itemize}
\end{frame}

%//end of section "Motivation"

\section{Konzept}

\begin{frame}{Inhaltsverzeichnis}
	\tableofcontents[currentsection, hideallsubsections]
\end{frame}


%-----Konzept-----

\begin{frame}{Konzept}{Was brauchen wir?}
	\begin{itemize}[<+->]
		\item Freien Code
		\item Fallback-Lösungen für geschwächte/zerstörte Infrastruktur
		\item Simple und leicht zu implementierende, aber zugleich starke Kryptographie
		\item Zu gewährleistende Anonymität
		\item Simple-To-Use-Client mit vielen Features
	\end{itemize}
\end{frame}

%-----Überleitung-----

\begin{frame}{Konzept}{Wer kann uns das Alles geben?}
	\begin{center}
		\Huge ?
	\end{center}
\end{frame}


%\\end of section "Konzept"

\section{Bramble}

\begin{frame}{Inhaltsverzeichnis}
	\tableofcontents[currentsection, hideallsubsections]
\end{frame}


%Bramble
\begin{frame}{Bramble}{Das Arbeitspferd unter der Decke}
	\begin{itemize}[<+->]
	\item Abstraktionsschicht unter den eigentlichen Clients
	\item Stellt P2P-Verbindungen her
	\item Beinhaltet alle Crypto
	\item Vermittelt die Nachrichten
	\item Einfaches Einbinden in eigene Projekte (Clients)
	\end{itemize}
\end{frame}

%Verbindungen
\begin{frame}{Bramble}{Verbindungen}
	\begin{itemize}[<+->]
		\item LAN
		\item WLAN
		\item TOR
		\item Bluetooth
		\item Speicher
	\end{itemize}
\end{frame}

%Das Internet
\begin{frame}{Bramble}{Internet?}
	\begin{itemize}[<+->]
		\item Das Internet ist tot,
		\item es lebe das Internet!
		\item TOR ist die einzige Verbindung welche über das Internet kommuniziert
		\item Sonstige Verbindungen können in beliebigen Netzen funktionieren
	\end{itemize}
\end{frame}

\begin{frame}{Bramble}{Irgendwas fehlt doch noch?!}
	\begin{center}
		\Huge ?
	\end{center}
\end{frame}


%\\end of section "Bramble"

\section{Briar}


%Briar
\begin{frame}{Briar}{Ein Simple-To-Use-Client}
	\begin{center}
		\Huge Demo \\[20pt]
		\includegraphics[scale=0.7]{res/briar_apk_dl.png}
	\end{center}
\end{frame}

%\\end of section "Briar"

\begin{frame}
\vfill
\begin{center}\begin{Huge}Vielen Dank \\
für eure Aufmerksamkeit. \\[50pt]
**Free Discussion**\end{Huge}\vfill
\end{center}
\vfill
\end{frame}

\end{document}